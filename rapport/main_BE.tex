\documentclass[11pt]{article}
\usepackage[utf8]{inputenc}
\usepackage[french]{babel}
\usepackage{amsmath, amssymb}
\usepackage{graphicx}
\usepackage{geometry}
\usepackage{hyperref}
\usepackage{fancyhdr}
\usepackage{enumitem}
\usepackage{titlesec}
\usepackage{lmodern}
\geometry{margin=2.5cm}

\title{Modélisation de la parole par LPC \\[0.5em]
\large RAPPORT DE BE}
\author{Raphaël ADDI, Mehdi DHAHRI \\
INSA Toulouse – Département 3MIC \\
Encadrant : Pascal Noble}
\date{17 avril 2025}

\begin{document}

\maketitle
\thispagestyle{empty}
\newpage

\tableofcontents
\newpage

\section{Introduction}

Ce pré-rapport présente l’avancement du BE sur la modélisation de la parole à l’aide d’un modèle LPC (Linear Predictive Coding). L’objectif est d’analyser, synthétiser, et manipuler des signaux de parole à des fins de compression, reconnaissance et transformation vocale.

\section{Modèle mathématique de la production vocale}

Un signal vocal \( x[n] \) est modélisé comme la sortie d’un filtre AR (Auto-Régressif) d’ordre \( p \) :

\[
x[n] = \sum_{k=1}^{p} a_k x[n-k] + e[n]
\]

où \( e[n] \) est le signal d’excitation :
\begin{itemize}
    \item bruit blanc pour les sons non voisés,
    \item train d’impulsions (Dirac) pour les sons voisés.
\end{itemize}

\section{Analyse LPC : du signal à l'encodage}

\subsection{Découpage en trames (fenêtrage)}

Le signal est découpé en trames de \( n_w = 240 \) échantillons (soit 30 ms à 8kHz) avec un recouvrement \( R = 0.5 \).

\begin{itemize}
    \item \texttt{CreateTrame} : génère les trames pondérées par une fenêtre de Hann.
    \item \texttt{AddTrame} : reconstruit le signal par Overlap-Add.
\end{itemize}

\subsection{Estimation des coefficients LPC}

Chaque trame \( x \) est modélisée comme une combinaison linéaire de ses valeurs passées. On cherche un vecteur \( a = (a_1, \ldots, a_p)^T \) tel que :

\[
x[t] \approx \sum_{k=1}^{p} a_k x[t - k] + e[t] \quad \text{pour } t = p, \ldots, n-1
\]

On forme alors un système surdéterminé \( Xa = b \), où :

\[
X =
\begin{bmatrix}
x[p-1] & x[p-2] & \dots & x[0] \\
x[p] & x[p-1] & \dots & x[1] \\
\vdots & \vdots & \ddots & \vdots \\
x[n-2] & x[n-3] & \dots & x[n-p-1]
\end{bmatrix}, \quad
b =
\begin{bmatrix}
x[p] \\
x[p+1] \\
\vdots \\
x[n-1]
\end{bmatrix}
\]

Si \( p < nb \), alors le système est surdéterminé, ce qui signifie qu’il y a plus d’équations que d’inconnues. On résout ce système au sens des moindres carrés :

\[
a = (X^T X)^{-1} X^T b
\]

L’erreur \( e = b - Xa \) est modélisée comme un bruit blanc, hypothèse raisonnable dans le cas voisé. Elle peut cependant être testée dans le cas non voisé.

La variance du bruit est estimée par :

\[
\sigma^2 = \text{Var}(b - Xa)
\]

La fonction \texttt{EncodeLPC} applique ce traitement à chaque trame du signal et retourne :
\begin{itemize}
    \item \( A \in \mathbb{R}^{p \times nb} \), matrice contenant les coefficients de chaque trame ;
    \item \( G \in \mathbb{R}^{nb} \), vecteur des variances de chaque trame.
\end{itemize}

\subsection{Erreur de reconstruction selon l'ordre \( p \)}

On évalue la fidélité de la reconstruction en calculant l’erreur entre le signal original et le signal reconstruit à l’aide du modèle LPC pour différents ordres \( p \).

Les métriques utilisées sont :
\begin{itemize}
    \item La norme \( L_1 \), moyenne des valeurs absolues des erreurs :
    \[
    \|x - \hat{x}\|_{L_1} = \frac{1}{N} \sum_{n=0}^{N-1} |x[n] - \hat{x}[n]|
    \]
    \item La norme \( L_2 \), racine carrée de la moyenne des carrés des erreurs :
    \[
    \|x - \hat{x}\|_{L_2} = \sqrt{ \frac{1}{N} \sum_{n=0}^{N-1} (x[n] - \hat{x}[n])^2 }
    \]
\end{itemize}

Ces erreurs sont calculées trame par trame, puis moyennées. Le nombre de trames est noté \( nb \).

\begin{center}
\begin{tabular}{|c|c|c|}
\hline
Ordre \( p \) & Erreur \( L_1 \) & Erreur \( L_2 \) \\
\hline
6 & 0.0807 & 0.1379 \\
12 & 0.0797 & 0.1351 \\
24 & 0.0788 & 0.1330 \\
\hline
\end{tabular}
\end{center}

\section{Synthèse LPC}

La fonction \texttt{DecodeLPC} utilise les coefficients \( A \) et les variances \( G \) pour reconstruire un signal à partir d’un bruit blanc normalisé, via la fonction \texttt{RunAR}.

Afin d’éviter toute saturation du signal reconstruit :
\begin{itemize}
    \item un \textit{clipping} à l’intervalle \([-1, 1]\) est appliqué ;
    \item une normalisation à 90\,\% de l’amplitude maximale suit.
\end{itemize}

\section{Synthèse croisée}

La synthèse croisée consiste à appliquer les coefficients \( A \) d’un signal (modulation) sur les gains \( G \) d’un autre signal (porteur). Cela permet de transférer les caractéristiques spectrales d’une voix à une autre.

Fonctions principales :
\begin{itemize}
    \item \texttt{EncodeLPC} pour calculer les coefficients et variances ;
    \item \texttt{RunAR} pour générer chaque trame ;
    \item \texttt{AddTrame} pour reconstruire le signal complet.
\end{itemize}

\section{Vers la reconnaissance vocale}

\subsection{Voisement et pitch}

On commence par détecter les trames voisées (voisement) grâce à la fonction \texttt{Voisement()}, fondée sur l’autocorrélation.

La fonction \texttt{EstimationPitch()} estime ensuite la fréquence fondamentale (pitch) des trames voisées.

\subsection{Excitation et synthèse améliorée}

Selon le voisement, l’excitation est :
\begin{itemize}
    \item un train d’impulsions pour les trames voisées ;
    \item du bruit blanc pour les trames non voisées.
\end{itemize}

La fonction \texttt{ConstructionExcitation()} crée ce signal d’excitation, qui est ensuite filtré par \texttt{SyntheseLPC()} pour obtenir un signal de parole synthétique.

\subsection{Paramètres spectraux et dictionnaire}

Pour chaque trame du signal, on extrait les \texttt{formants}, c’est-à-dire :
\begin{itemize}
\item les 	\texttt{positions des pics} du spectre LPC (fréquences caractéristiques),
\item les \texttt{amplitudes} de ces pics (intensité spectrale).
\end{itemize}

Ces caractéristiques sont à la fois \texttt{stables} et 	\texttt{discriminantes} entre chaque sons (notamment pour les voyelles). Elles permettent donc de reconnaître ou différencier les sons parlés.\newline

On les compare ensuite aux formants extraits d’un \texttt{dictionnaire de sons connus} (préconstruit à partir de fichiers \texttt{.wav} il y en a environ 70 afin d'avoir un maximum de comparaison possible si on voulait être plus précis ils nous en aurait fallut beaucoup plus) a l'aide d'une mesure de distance. Le son du dictionnaire ayant les formants les plus proches est considéré comme le plus ressemblant. \newline

La fonction \texttt{EstimationSon(...)} permet d’automatiser ce processus :
\begin{itemize}
\item elle découpe le signal inconnu en trames,
\item extrait les formants de chaque trame,
\item les compare à ceux du dictionnaire,
\item et retourne une séquence d’indices correspondant au son le plus proche pour chaque trame.\newline
\end{itemize}

Cependant, des erreurs ponctuelles peuvent apparaître à cause du bruit ou de variations locales. Pour améliorer la cohérence temporelle, on utilise alors la fonction \texttt{CorrectionSon(...)}, qui :

\begin{itemize}
    \item détecte les trames isolées incohérentes,
    \item les remplace par le son dominant dans les trames voisines
\end{itemize}
Cette post-correction permet de lisser la reconnaissance dans le temps et d’obtenir un résultat plus fiable.


\section{Conclusion et perspectives}

Le modèle LPC est efficace pour compresser, synthétiser et transformer des signaux vocaux. Il permet également d’extraire des caractéristiques utiles pour la reconnaissance vocale.

Perspectives :
\begin{itemize}
    \item Améliorer la robustesse en environnement bruité ;
    \item Tester d'autres modèles d'excitation plus réalistes ;
    \item Développer une reconnaissance multi-locuteur avec apprentissage.
\end{itemize}

\section{Annexe}
Nous vous avons mis les fichiers \texttt{.wav} et les graphiques sur un GitHub afin que vous puissier tout retrouver.
\end{Annexe}

\addcontentsline{toc}{section}{Références}
\begin{thebibliography}{9}

\bibitem{bergounioux}
Maitine Bergounioux, 
\textit{Mathématiques pour le Traitement du Signal}, 
Éditions Ellipses, 2004.

\bibitem{kim}
Hyung-Suk Kim, 
\textit{Linear Predictive Coding is All-Pole Resonance Modeling}, 
arXiv preprint arXiv:1606.04035, 2016.

\end{Références}

\end{document}